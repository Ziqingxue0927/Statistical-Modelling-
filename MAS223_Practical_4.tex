% Options for packages loaded elsewhere
\PassOptionsToPackage{unicode}{hyperref}
\PassOptionsToPackage{hyphens}{url}
%
\documentclass[
]{article}
\usepackage{amsmath,amssymb}
\usepackage{iftex}
\ifPDFTeX
  \usepackage[T1]{fontenc}
  \usepackage[utf8]{inputenc}
  \usepackage{textcomp} % provide euro and other symbols
\else % if luatex or xetex
  \usepackage{unicode-math} % this also loads fontspec
  \defaultfontfeatures{Scale=MatchLowercase}
  \defaultfontfeatures[\rmfamily]{Ligatures=TeX,Scale=1}
\fi
\usepackage{lmodern}
\ifPDFTeX\else
  % xetex/luatex font selection
\fi
% Use upquote if available, for straight quotes in verbatim environments
\IfFileExists{upquote.sty}{\usepackage{upquote}}{}
\IfFileExists{microtype.sty}{% use microtype if available
  \usepackage[]{microtype}
  \UseMicrotypeSet[protrusion]{basicmath} % disable protrusion for tt fonts
}{}
\makeatletter
\@ifundefined{KOMAClassName}{% if non-KOMA class
  \IfFileExists{parskip.sty}{%
    \usepackage{parskip}
  }{% else
    \setlength{\parindent}{0pt}
    \setlength{\parskip}{6pt plus 2pt minus 1pt}}
}{% if KOMA class
  \KOMAoptions{parskip=half}}
\makeatother
\usepackage{xcolor}
\usepackage[margin=1in]{geometry}
\usepackage{color}
\usepackage{fancyvrb}
\newcommand{\VerbBar}{|}
\newcommand{\VERB}{\Verb[commandchars=\\\{\}]}
\DefineVerbatimEnvironment{Highlighting}{Verbatim}{commandchars=\\\{\}}
% Add ',fontsize=\small' for more characters per line
\usepackage{framed}
\definecolor{shadecolor}{RGB}{248,248,248}
\newenvironment{Shaded}{\begin{snugshade}}{\end{snugshade}}
\newcommand{\AlertTok}[1]{\textcolor[rgb]{0.94,0.16,0.16}{#1}}
\newcommand{\AnnotationTok}[1]{\textcolor[rgb]{0.56,0.35,0.01}{\textbf{\textit{#1}}}}
\newcommand{\AttributeTok}[1]{\textcolor[rgb]{0.13,0.29,0.53}{#1}}
\newcommand{\BaseNTok}[1]{\textcolor[rgb]{0.00,0.00,0.81}{#1}}
\newcommand{\BuiltInTok}[1]{#1}
\newcommand{\CharTok}[1]{\textcolor[rgb]{0.31,0.60,0.02}{#1}}
\newcommand{\CommentTok}[1]{\textcolor[rgb]{0.56,0.35,0.01}{\textit{#1}}}
\newcommand{\CommentVarTok}[1]{\textcolor[rgb]{0.56,0.35,0.01}{\textbf{\textit{#1}}}}
\newcommand{\ConstantTok}[1]{\textcolor[rgb]{0.56,0.35,0.01}{#1}}
\newcommand{\ControlFlowTok}[1]{\textcolor[rgb]{0.13,0.29,0.53}{\textbf{#1}}}
\newcommand{\DataTypeTok}[1]{\textcolor[rgb]{0.13,0.29,0.53}{#1}}
\newcommand{\DecValTok}[1]{\textcolor[rgb]{0.00,0.00,0.81}{#1}}
\newcommand{\DocumentationTok}[1]{\textcolor[rgb]{0.56,0.35,0.01}{\textbf{\textit{#1}}}}
\newcommand{\ErrorTok}[1]{\textcolor[rgb]{0.64,0.00,0.00}{\textbf{#1}}}
\newcommand{\ExtensionTok}[1]{#1}
\newcommand{\FloatTok}[1]{\textcolor[rgb]{0.00,0.00,0.81}{#1}}
\newcommand{\FunctionTok}[1]{\textcolor[rgb]{0.13,0.29,0.53}{\textbf{#1}}}
\newcommand{\ImportTok}[1]{#1}
\newcommand{\InformationTok}[1]{\textcolor[rgb]{0.56,0.35,0.01}{\textbf{\textit{#1}}}}
\newcommand{\KeywordTok}[1]{\textcolor[rgb]{0.13,0.29,0.53}{\textbf{#1}}}
\newcommand{\NormalTok}[1]{#1}
\newcommand{\OperatorTok}[1]{\textcolor[rgb]{0.81,0.36,0.00}{\textbf{#1}}}
\newcommand{\OtherTok}[1]{\textcolor[rgb]{0.56,0.35,0.01}{#1}}
\newcommand{\PreprocessorTok}[1]{\textcolor[rgb]{0.56,0.35,0.01}{\textit{#1}}}
\newcommand{\RegionMarkerTok}[1]{#1}
\newcommand{\SpecialCharTok}[1]{\textcolor[rgb]{0.81,0.36,0.00}{\textbf{#1}}}
\newcommand{\SpecialStringTok}[1]{\textcolor[rgb]{0.31,0.60,0.02}{#1}}
\newcommand{\StringTok}[1]{\textcolor[rgb]{0.31,0.60,0.02}{#1}}
\newcommand{\VariableTok}[1]{\textcolor[rgb]{0.00,0.00,0.00}{#1}}
\newcommand{\VerbatimStringTok}[1]{\textcolor[rgb]{0.31,0.60,0.02}{#1}}
\newcommand{\WarningTok}[1]{\textcolor[rgb]{0.56,0.35,0.01}{\textbf{\textit{#1}}}}
\usepackage{graphicx}
\makeatletter
\def\maxwidth{\ifdim\Gin@nat@width>\linewidth\linewidth\else\Gin@nat@width\fi}
\def\maxheight{\ifdim\Gin@nat@height>\textheight\textheight\else\Gin@nat@height\fi}
\makeatother
% Scale images if necessary, so that they will not overflow the page
% margins by default, and it is still possible to overwrite the defaults
% using explicit options in \includegraphics[width, height, ...]{}
\setkeys{Gin}{width=\maxwidth,height=\maxheight,keepaspectratio}
% Set default figure placement to htbp
\makeatletter
\def\fps@figure{htbp}
\makeatother
\setlength{\emergencystretch}{3em} % prevent overfull lines
\providecommand{\tightlist}{%
  \setlength{\itemsep}{0pt}\setlength{\parskip}{0pt}}
\setcounter{secnumdepth}{-\maxdimen} % remove section numbering
\ifLuaTeX
  \usepackage{selnolig}  % disable illegal ligatures
\fi
\IfFileExists{bookmark.sty}{\usepackage{bookmark}}{\usepackage{hyperref}}
\IfFileExists{xurl.sty}{\usepackage{xurl}}{} % add URL line breaks if available
\urlstyle{same}
\hypersetup{
  pdftitle={Analysis of UK Healthcare Performance in 2011},
  hidelinks,
  pdfcreator={LaTeX via pandoc}}

\title{Analysis of UK Healthcare Performance in 2011}
\author{}
\date{\vspace{-2.5em}}

\begin{document}
\maketitle

\hypertarget{introduction}{%
\section{Introduction}\label{introduction}}

In 2011, UK's former Health Secretary Andrew Lansley criticized that
UK's previous health-care provision did not get the results compared
with other European countries. This report will assess the performance
of UK healthcare in 2011 using the data from WHO, which includes life
expectancy, expenditure, tobacco consumption, alcohol consumption, and
obesity rates in OECD countries. After comparing the 2011 performance of
UK healthcare with healthcare in other OECD countries, not enough
evidence has been found to support Andrew Lansley's claim.

\hypertarget{results}{%
\section{Results}\label{results}}

In this report, the indicator of health experience is set to be life
expectancy. There are several variables which may affect health
outcomes: expenditure, tobacco consumption, alcohol consumption, and
obesity rates. The relationship between those variables and life
expectancy in OECD countries except the UK will be analyzed first. Then
UK's data will be assessed to see whether it fits that relation well.

To visualize the data points, scatter plots of life expectancy against
each independent variable (tobacco consumption, expenditure, alcohol
consumption, and obesity rates) are produced below:

\includegraphics{"2.jpg"}

First, the model being chosen for the dataset is:
\[M_f: y_i=\beta_0+\beta_1v_i+\beta_2w_i+\beta_3x_i+\beta_4z_i+\epsilon_i,\]
where \(y_i\) is the life expectancy in years, \(v_i\) is percentage of
tobacco user among adults, \(w_i\) is the total expenditure per person
on health, \(x_i\) is alcohol consumption among adults, and \(z_i\) is
percentage of adults classified as obese. It is assumed that
\(\epsilon_i \sim N(0,\sigma^2)\) \textbf{i.i.d.}.

To check the assumption \(\epsilon_i \sim N(0,\sigma^2)\), a histogram
and a Q-Q plot for the standardized residuals of the fitted model are
produced below:

\includegraphics{MAS223_Practical_4_files/figure-latex/unnamed-chunk-2-1.pdf}

In addition, the KS test gives the p-value 0.5636, which is large. These
all suggest that the normality assumption is appropriate.

However, when testing whether \(\epsilon_i\) are \textbf{i.i.d.}, it is
found that the original model might need transformation. The plots below
shows, respectively, the standardized residuals against the fitted
values, and scatter plots of life expectancy against tobacco
consumption, expenditure, log(expenditure), alcohol consumption and
obesity rates.

\includegraphics{"1.jpg"}

It seems that the variance of residuals decreases as \(\hat y_i\)
increases. By comparing the third and fourth plot, we find that there
might be a non-linear relationship between expenditure and life
expectancy, which may be better modeled on the log scale.

In this case the full model now becomes
\[M_t: y_i=\beta_0+\beta_1v_i+\beta_2\log(w_i)+\beta_3x_i+\beta_4z_i+\epsilon_i.\]
The plot of the standardized residuals against the fitted values is
produced below, which suggests that the assumption of a constant
variance might be valid in the transformed model. In addition, the KS
test on this model gives the p-value 0.6314, which again suggest that
the normality assumption is appropriate.

\includegraphics{MAS223_Practical_4_files/figure-latex/unnamed-chunk-3-1.pdf}

We can then proceed to identify the most suitable model for the data.
First, we test the null hypothesis \[H_0: \beta_4=0.\] Then the reduced
model is
\[M_1: y_i=\beta_0+\beta_1v_i+\beta_2\log(w_i)+\beta_3x_i+\epsilon_i.\]After
the F-test is applied, the F-statistic is 5.4449 and the p-value is
0.02797 so there is enough evidence against the null hypothesis. So we
reject the null hypothesis.

Then we test the null hypothesis \[H_0: \beta_3=0,\] which claims that
the life expectancy is independent of alcohol consumption. Then the
reduced model is
\[M_2: y_i=\beta_0+\beta_1v_i+\beta_2\log(w_i)+\beta_4z_i+\epsilon_i.\]
The result gives F-statistic 5.8591 and p-value 0.02309 so there is
enough evidence against the null hypothesis. So we reject the null
hypothesis.

After that we test the null hypothesis \[H_0: \beta_2=0.\] Then the
reduced model is
\[M_3: y_i=\beta_0+\beta_1v_i+\beta_3x_i+\beta_4z_i+\epsilon_i.\] The
result gives F-statistic 19.866 and p-value 0.0001522 so there is strong
evidence against the null hypothesis. So we reject the null hypothesis.

Finally, we test the null hypothesis \[H_0: \beta_1=0.\] Then the
reduced model is
\[M_4: y_i=\beta_0+\beta_2\log(w_i)+\beta_3x_i+\beta_4z_i+\epsilon_i.\]
The result gives F-statistic 0.8116 and p-value 0.3762 so there is not
enough evidence against the null hypothesis. So there is no evidence to
reject the model \(M_4\) in favor the full model \(M_t\).

Then we can treat \(M_4\) as the full model, and test whether
\(\beta_2=0\), \(\beta_3=0\) and \(\beta_4=0\) are true or not. The
p-values of these F-tests give 4.622e-05, 0.007052 and 0.03432
respectively, meaning that there is evidence against all the null
hypotheses.

Therefore, we may choose
\[M_4: y_i=\beta_0+\beta_2\log(w_i)+\beta_3x_i+\beta_4z_i+\epsilon_i\]
as the most suitable model for the data

Now we can put the data of the UK into our model to check whether the
relationship between life expectancy and other variables in the UK is
similar to the relationship in other OECD countries, so that we can know
whether the 2011 performance of UK healthcare is close to healthcare in
other OECD countries.

Given data about health expenditure, alcohol consumption, and obesity
rates in the UK, we can predict the life expectancy in the UK according
to our model. The 95\% prediction interval for life expectancy in the UK
is \([75.36987,82.19234]\), and the actual life expectancy is 80 which
is in the prediction interval. Therefore, there is not enough evidence
to show that there exists significant difference between the 2011
performance of UK healthcare and healthcare in other OECD countries.

\hypertarget{conclusion}{%
\section{Conclusion}\label{conclusion}}

To conclude, firstly, it has been found that life expectancy depends
mostly on (the logarithm of) health expenditure, alcohol consumption,
and obesity rates. In addition, a linear model has been fitted in this
report to describe the relationship between the life expectancy and
those variables in OECD countries except the UK. Furthermore, given data
about health expenditure, alcohol consumption, and obesity rates in the
UK, a prediction of life expectancy as well as the prediction interval
has been produced, and this prediction fits the real situation well,
which suggests that there is not enough evidence to support Andrew
Lansley's claim.

\hypertarget{appendix-r-commands}{%
\section{Appendix: R commands}\label{appendix-r-commands}}

\begin{Shaded}
\begin{Highlighting}[]
\FunctionTok{library}\NormalTok{(MASS)}
\FunctionTok{load}\NormalTok{(}\StringTok{"MAS223.RData"}\NormalTok{)}
\CommentTok{\# Choose the data except the data in the UK.}
\NormalTok{who\_exUK }\OtherTok{\textless{}{-}}\NormalTok{ who[who}\SpecialCharTok{$}\NormalTok{country }\SpecialCharTok{!=} \StringTok{"United Kingdom"}\NormalTok{,]}
\FunctionTok{attach}\NormalTok{(who\_exUK)}
\end{Highlighting}
\end{Shaded}

\begin{Shaded}
\begin{Highlighting}[]
\CommentTok{\# Scatter plots of life expectancy against each independent variable.}
\FunctionTok{par}\NormalTok{(}\AttributeTok{mfrow=}\FunctionTok{c}\NormalTok{(}\DecValTok{1}\NormalTok{,}\DecValTok{4}\NormalTok{))}
\FunctionTok{plot}\NormalTok{(tobacco,life,}\AttributeTok{xlab =} \StringTok{"Tobacco consumption"}\NormalTok{,}\AttributeTok{ylab =} \StringTok{"Life expectancy"}\NormalTok{)}
\FunctionTok{plot}\NormalTok{(expenditure,life,}\AttributeTok{xlab =} \StringTok{"Expenditure"}\NormalTok{,}\AttributeTok{ylab =} \StringTok{"Life expectancy"}\NormalTok{)}
\FunctionTok{plot}\NormalTok{(alcohol,life,}\AttributeTok{xlab =} \StringTok{"Alcohol consumption"}\NormalTok{,}\AttributeTok{ylab =} \StringTok{"Life expectancy"}\NormalTok{)}
\FunctionTok{plot}\NormalTok{(obesity,life,}\AttributeTok{xlab =} \StringTok{"Obesity rates"}\NormalTok{,}\AttributeTok{ylab =} \StringTok{"Life expectancy"}\NormalTok{)}
\end{Highlighting}
\end{Shaded}

\begin{Shaded}
\begin{Highlighting}[]
\CommentTok{\# The original full model.}
\NormalTok{lm\_f }\OtherTok{\textless{}{-}} \FunctionTok{lm}\NormalTok{(life}\SpecialCharTok{\textasciitilde{}}\NormalTok{tobacco }\SpecialCharTok{+}\NormalTok{ expenditure }\SpecialCharTok{+}\NormalTok{ alcohol }\SpecialCharTok{+}\NormalTok{ obesity)}

\CommentTok{\# The histogram and Q{-}Q plot for the standardized residuals and KS test.}
\FunctionTok{par}\NormalTok{(}\AttributeTok{mfrow=}\FunctionTok{c}\NormalTok{(}\DecValTok{1}\NormalTok{,}\DecValTok{2}\NormalTok{))}
\NormalTok{evals }\OtherTok{\textless{}{-}} \FunctionTok{stdres}\NormalTok{(lm\_f)}
\FunctionTok{hist}\NormalTok{(evals,}\AttributeTok{xlab =} \StringTok{"e"}\NormalTok{,}\AttributeTok{ylab =} \StringTok{""}\NormalTok{)}
\FunctionTok{qqnorm}\NormalTok{(evals)}
\FunctionTok{abline}\NormalTok{(}\DecValTok{0}\NormalTok{,}\DecValTok{1}\NormalTok{)}
\end{Highlighting}
\end{Shaded}

\begin{Shaded}
\begin{Highlighting}[]
\FunctionTok{ks.test}\NormalTok{(evals,pnorm,}\DecValTok{0}\NormalTok{,}\DecValTok{1}\NormalTok{)}
\end{Highlighting}
\end{Shaded}

\begin{verbatim}
## 
##  One-sample Kolmogorov-Smirnov test
## 
## data:  evals
## D = 0.13872, p-value = 0.5636
## alternative hypothesis: two-sided
\end{verbatim}

\begin{Shaded}
\begin{Highlighting}[]
\CommentTok{\# The plot of the standardized residuals against the fitted values as well as }
\CommentTok{\# scatter plots of life expectancy against each independent variable to test i.i.d.}
\FunctionTok{par}\NormalTok{(}\AttributeTok{mfrow=}\FunctionTok{c}\NormalTok{(}\DecValTok{1}\NormalTok{,}\DecValTok{6}\NormalTok{))}
\FunctionTok{plot}\NormalTok{(}\FunctionTok{fitted}\NormalTok{(lm\_f),evals)}
\FunctionTok{plot}\NormalTok{(tobacco,life,}\AttributeTok{xlab =} \StringTok{"Tobacco consumption"}\NormalTok{,}\AttributeTok{ylab =} \StringTok{"Life expectancy"}\NormalTok{)}
\FunctionTok{plot}\NormalTok{(expenditure,life,}\AttributeTok{xlab =} \StringTok{"Expenditure"}\NormalTok{,}\AttributeTok{ylab =} \StringTok{"Life expectancy"}\NormalTok{)}
\FunctionTok{plot}\NormalTok{(}\FunctionTok{log}\NormalTok{(expenditure),life,}\AttributeTok{xlab =} \StringTok{"log(Expenditure)"}\NormalTok{,}\AttributeTok{ylab =} \StringTok{"Life expectancy"}\NormalTok{)}
\FunctionTok{plot}\NormalTok{(alcohol,life,}\AttributeTok{xlab =} \StringTok{"Alcohol consumption"}\NormalTok{,}\AttributeTok{ylab =} \StringTok{"Life expectancy"}\NormalTok{)}
\FunctionTok{plot}\NormalTok{(obesity,life,}\AttributeTok{xlab =} \StringTok{"Obesity rates"}\NormalTok{,}\AttributeTok{ylab =} \StringTok{"Life expectancy"}\NormalTok{)}
\end{Highlighting}
\end{Shaded}

\begin{Shaded}
\begin{Highlighting}[]
\CommentTok{\# Transformed full model and plot of its standardized residuals against the fitted values}
\NormalTok{lm\_t }\OtherTok{\textless{}{-}} \FunctionTok{lm}\NormalTok{(life}\SpecialCharTok{\textasciitilde{}}\NormalTok{tobacco }\SpecialCharTok{+} \FunctionTok{log}\NormalTok{(expenditure) }\SpecialCharTok{+}\NormalTok{ alcohol }\SpecialCharTok{+}\NormalTok{ obesity)}
\NormalTok{evals1 }\OtherTok{\textless{}{-}} \FunctionTok{stdres}\NormalTok{(lm\_t)}
\FunctionTok{plot}\NormalTok{(}\FunctionTok{fitted}\NormalTok{(lm\_t),evals1,}\AttributeTok{xlab =} \StringTok{"Fitted values"}\NormalTok{,}\AttributeTok{ylab =} \StringTok{"e"}\NormalTok{)}
\end{Highlighting}
\end{Shaded}

\begin{Shaded}
\begin{Highlighting}[]
\FunctionTok{ks.test}\NormalTok{(evals1,pnorm,}\DecValTok{0}\NormalTok{,}\DecValTok{1}\NormalTok{)}
\end{Highlighting}
\end{Shaded}

\begin{verbatim}
## 
##  One-sample Kolmogorov-Smirnov test
## 
## data:  evals1
## D = 0.13135, p-value = 0.6314
## alternative hypothesis: two-sided
\end{verbatim}

\begin{Shaded}
\begin{Highlighting}[]
\CommentTok{\# F{-}tests.}
\NormalTok{lm1 }\OtherTok{\textless{}{-}} \FunctionTok{lm}\NormalTok{(life}\SpecialCharTok{\textasciitilde{}}\NormalTok{tobacco }\SpecialCharTok{+} \FunctionTok{log}\NormalTok{(expenditure) }\SpecialCharTok{+}\NormalTok{ alcohol)}
\FunctionTok{anova}\NormalTok{(lm1,lm\_t)}
\end{Highlighting}
\end{Shaded}

\begin{verbatim}
## Analysis of Variance Table
## 
## Model 1: life ~ tobacco + log(expenditure) + alcohol
## Model 2: life ~ tobacco + log(expenditure) + alcohol + obesity
##   Res.Df    RSS Df Sum of Sq      F  Pr(>F)  
## 1     26 76.484                              
## 2     25 62.805  1    13.679 5.4449 0.02797 *
## ---
## Signif. codes:  0 '***' 0.001 '**' 0.01 '*' 0.05 '.' 0.1 ' ' 1
\end{verbatim}

\begin{Shaded}
\begin{Highlighting}[]
\NormalTok{lm2 }\OtherTok{\textless{}{-}} \FunctionTok{lm}\NormalTok{(life}\SpecialCharTok{\textasciitilde{}}\NormalTok{tobacco }\SpecialCharTok{+} \FunctionTok{log}\NormalTok{(expenditure) }\SpecialCharTok{+}\NormalTok{ obesity)}
\FunctionTok{anova}\NormalTok{(lm2,lm\_t)}
\end{Highlighting}
\end{Shaded}

\begin{verbatim}
## Analysis of Variance Table
## 
## Model 1: life ~ tobacco + log(expenditure) + obesity
## Model 2: life ~ tobacco + log(expenditure) + alcohol + obesity
##   Res.Df    RSS Df Sum of Sq      F  Pr(>F)  
## 1     26 77.525                              
## 2     25 62.805  1    14.719 5.8591 0.02309 *
## ---
## Signif. codes:  0 '***' 0.001 '**' 0.01 '*' 0.05 '.' 0.1 ' ' 1
\end{verbatim}

\begin{Shaded}
\begin{Highlighting}[]
\NormalTok{lm3 }\OtherTok{\textless{}{-}} \FunctionTok{lm}\NormalTok{(life}\SpecialCharTok{\textasciitilde{}}\NormalTok{tobacco }\SpecialCharTok{+}\NormalTok{ alcohol }\SpecialCharTok{+}\NormalTok{ obesity)}
\FunctionTok{anova}\NormalTok{(lm3,lm\_t)}
\end{Highlighting}
\end{Shaded}

\begin{verbatim}
## Analysis of Variance Table
## 
## Model 1: life ~ tobacco + alcohol + obesity
## Model 2: life ~ tobacco + log(expenditure) + alcohol + obesity
##   Res.Df     RSS Df Sum of Sq      F    Pr(>F)    
## 1     26 112.714                                  
## 2     25  62.805  1    49.908 19.866 0.0001522 ***
## ---
## Signif. codes:  0 '***' 0.001 '**' 0.01 '*' 0.05 '.' 0.1 ' ' 1
\end{verbatim}

\begin{Shaded}
\begin{Highlighting}[]
\NormalTok{lm4 }\OtherTok{\textless{}{-}} \FunctionTok{lm}\NormalTok{(life}\SpecialCharTok{\textasciitilde{}}\FunctionTok{log}\NormalTok{(expenditure) }\SpecialCharTok{+}\NormalTok{ alcohol }\SpecialCharTok{+}\NormalTok{ obesity)}
\FunctionTok{anova}\NormalTok{(lm4,lm\_t)}
\end{Highlighting}
\end{Shaded}

\begin{verbatim}
## Analysis of Variance Table
## 
## Model 1: life ~ log(expenditure) + alcohol + obesity
## Model 2: life ~ tobacco + log(expenditure) + alcohol + obesity
##   Res.Df    RSS Df Sum of Sq      F Pr(>F)
## 1     26 64.844                           
## 2     25 62.805  1    2.0389 0.8116 0.3762
\end{verbatim}

\begin{Shaded}
\begin{Highlighting}[]
\NormalTok{lm4\_1 }\OtherTok{\textless{}{-}} \FunctionTok{lm}\NormalTok{(life}\SpecialCharTok{\textasciitilde{}}\NormalTok{alcohol }\SpecialCharTok{+}\NormalTok{ obesity)}
\FunctionTok{anova}\NormalTok{(lm4\_1,lm4)}
\end{Highlighting}
\end{Shaded}

\begin{verbatim}
## Analysis of Variance Table
## 
## Model 1: life ~ alcohol + obesity
## Model 2: life ~ log(expenditure) + alcohol + obesity
##   Res.Df     RSS Df Sum of Sq      F    Pr(>F)    
## 1     27 124.220                                  
## 2     26  64.844  1    59.376 23.807 4.622e-05 ***
## ---
## Signif. codes:  0 '***' 0.001 '**' 0.01 '*' 0.05 '.' 0.1 ' ' 1
\end{verbatim}

\begin{Shaded}
\begin{Highlighting}[]
\NormalTok{lm4\_2 }\OtherTok{\textless{}{-}} \FunctionTok{lm}\NormalTok{(life}\SpecialCharTok{\textasciitilde{}}\FunctionTok{log}\NormalTok{(expenditure) }\SpecialCharTok{+}\NormalTok{ obesity)}
\FunctionTok{anova}\NormalTok{(lm4\_2,lm4)}
\end{Highlighting}
\end{Shaded}

\begin{verbatim}
## Analysis of Variance Table
## 
## Model 1: life ~ log(expenditure) + obesity
## Model 2: life ~ log(expenditure) + alcohol + obesity
##   Res.Df    RSS Df Sum of Sq      F   Pr(>F)   
## 1     27 86.185                                
## 2     26 64.844  1     21.34 8.5567 0.007052 **
## ---
## Signif. codes:  0 '***' 0.001 '**' 0.01 '*' 0.05 '.' 0.1 ' ' 1
\end{verbatim}

\begin{Shaded}
\begin{Highlighting}[]
\NormalTok{lm4\_3 }\OtherTok{\textless{}{-}} \FunctionTok{lm}\NormalTok{(life}\SpecialCharTok{\textasciitilde{}}\FunctionTok{log}\NormalTok{(expenditure) }\SpecialCharTok{+}\NormalTok{ alcohol)}
\FunctionTok{anova}\NormalTok{(lm4\_3,lm4)}
\end{Highlighting}
\end{Shaded}

\begin{verbatim}
## Analysis of Variance Table
## 
## Model 1: life ~ log(expenditure) + alcohol
## Model 2: life ~ log(expenditure) + alcohol + obesity
##   Res.Df    RSS Df Sum of Sq      F  Pr(>F)  
## 1     27 77.289                              
## 2     26 64.844  1    12.444 4.9897 0.03432 *
## ---
## Signif. codes:  0 '***' 0.001 '**' 0.01 '*' 0.05 '.' 0.1 ' ' 1
\end{verbatim}

\begin{Shaded}
\begin{Highlighting}[]
\CommentTok{\# The best model.}
\FunctionTok{summary}\NormalTok{(lm4)}
\end{Highlighting}
\end{Shaded}

\begin{verbatim}
## 
## Call:
## lm(formula = life ~ log(expenditure) + alcohol + obesity)
## 
## Residuals:
##     Min      1Q  Median      3Q     Max 
## -3.2978 -0.8281  0.3349  0.9540  2.6344 
## 
## Coefficients:
##                  Estimate Std. Error t value Pr(>|t|)    
## (Intercept)      61.63717    4.96803  12.407 2.00e-12 ***
## log(expenditure)  2.82413    0.57880   4.879 4.62e-05 ***
## alcohol          -0.29708    0.10156  -2.925  0.00705 ** 
## obesity          -0.09413    0.04214  -2.234  0.03432 *  
## ---
## Signif. codes:  0 '***' 0.001 '**' 0.01 '*' 0.05 '.' 0.1 ' ' 1
## 
## Residual standard error: 1.579 on 26 degrees of freedom
## Multiple R-squared:  0.6218, Adjusted R-squared:  0.5782 
## F-statistic: 14.25 on 3 and 26 DF,  p-value: 1.09e-05
\end{verbatim}

\begin{Shaded}
\begin{Highlighting}[]
\CommentTok{\# Make prediction and compute prediction interval}
\NormalTok{uk\_values }\OtherTok{\textless{}{-}}\NormalTok{ who[who}\SpecialCharTok{$}\NormalTok{country }\SpecialCharTok{==} \StringTok{"United Kingdom"}\NormalTok{,]}
\NormalTok{predictions.PI }\OtherTok{\textless{}{-}} \FunctionTok{data.frame}\NormalTok{(}\FunctionTok{predict.lm}\NormalTok{(lm4,uk\_values,}
                                        \AttributeTok{interval =} \StringTok{"prediction"}\NormalTok{, }
                                        \AttributeTok{level =} \FloatTok{0.95}\NormalTok{))}
\NormalTok{predictions.PI}
\end{Highlighting}
\end{Shaded}

\begin{verbatim}
##        fit      lwr      upr
## 30 78.7811 75.36987 82.19234
\end{verbatim}

\begin{Shaded}
\begin{Highlighting}[]
\NormalTok{uk\_values}\SpecialCharTok{$}\NormalTok{life}
\end{Highlighting}
\end{Shaded}

\begin{verbatim}
## [1] 80
\end{verbatim}

No brown m\&m's

\end{document}
